\chapter{Implementacja}

\section{Konfiguracja środowiska}
Do stworzenia bota użyto języka Python w wersji 3. W tym celu zainstalowano najnowszą wersję, którą można znaleźć na oficjalnej stronie: \texttt{https://www.python.org/}.
Aby ułatwić pracę w środowisku, dodatkowo zainstalowano managera paczek do Python’a - PIP. Instrukcje dotyczące manualnej instalacji można znaleźć na oficjalnej stronie: \texttt{https://pypi.org/project/pip/}.

\section{Konfiguracja biblioteki ChatterBot}
Instalacja biblioteki ChatterBot jest bardzo prosta przy użyciu PIP. Całą bibliotekę można zainstalować jedną komendą:

\begin{center}
pip install chatterbot
\end{center}

Zainstalowaną bibliotekę można sprawdzić przy pomocy komendy:

\begin{center}
python -m chatterbot --version
\end{center}

Terminal powinien zwrócić aktualną wersję.

\section{Implementacja chatbota}

\subsection{Uczenie}
\subsection{Baza wiedzy}
\subsection{Dialog bota z użytkownikiem}
