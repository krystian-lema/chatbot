\chapter{Wstęp}

\section{Cel projektu}
Celem projektu jest stworzenie chatbota, którego zadaniem jest wsparcie klienta w sklepie internetowym ze sprzętem komputerowym. Komunikacja ma odbywać się w trybie tekstowym. Zadaniem chatbota jest rozpoznanie pytania zadanego przez klienta sklepu internetowego oraz znalezienie odpowiedzi w bazie wiedzy. Wykorzystano również technikę uczenia maszynowego do poprawienia skuteczności chatbota.

\section{Założenia projektowe}
Chatbot ma za zadanie pomagać klientom w następujących tematach:
\begin{itemize}
\item Obsługa konta użytkownika (zmiana hasła, przypomnienie hasła, bezpieczeństwo danych)
\item Dostawa (czas, koszt, rodzaje)
\item Zasady zwrotów (ile czasu na zwrot, koszt zwrotu, sposób)
\item Dostępne metody płatności
\item Śledzenie statusu zamówienia
\item Zasady reklamacji
\item Aktualne promocje
\end{itemize}

\section{Użyte technologie}
Chatbot został napisany w języku programowania Python z użyciem biblioteki dedykowanej do tworzenia chatbotów ChatterBot. ChatterBot to silnik do tworzenia dialogów, który używa bazy wiedzy w postaci plików tekstowych oraz uczenia maszynowego do zwiększenia skuteczności znajdywania odpowiednich odpowiedzi.

\newpage

\section{Zasada działania chatbota}
Schemat działania:
\begin{enumerate}
\item Odczytanie danych
\item Przetwarzanie surowego tekstu
\item Tokenizacja tekstu
\item Usunięcie zbędnych znaków
\item Usunięcie mało znaczących słów
\item Sprowadzenie słów do podstawowych form (lematyzacja)
\item Parsowanie zależności
\item Generowanie odpowiedzi na podstawie bazy wiedzy
\end{enumerate}
